% Options for packages loaded elsewhere
\PassOptionsToPackage{unicode}{hyperref}
\PassOptionsToPackage{hyphens}{url}
%
\documentclass[
]{book}
\usepackage{amsmath,amssymb}
\usepackage{lmodern}
\usepackage{ifxetex,ifluatex}
\ifnum 0\ifxetex 1\fi\ifluatex 1\fi=0 % if pdftex
  \usepackage[T1]{fontenc}
  \usepackage[utf8]{inputenc}
  \usepackage{textcomp} % provide euro and other symbols
\else % if luatex or xetex
  \usepackage{unicode-math}
  \defaultfontfeatures{Scale=MatchLowercase}
  \defaultfontfeatures[\rmfamily]{Ligatures=TeX,Scale=1}
\fi
% Use upquote if available, for straight quotes in verbatim environments
\IfFileExists{upquote.sty}{\usepackage{upquote}}{}
\IfFileExists{microtype.sty}{% use microtype if available
  \usepackage[]{microtype}
  \UseMicrotypeSet[protrusion]{basicmath} % disable protrusion for tt fonts
}{}
\makeatletter
\@ifundefined{KOMAClassName}{% if non-KOMA class
  \IfFileExists{parskip.sty}{%
    \usepackage{parskip}
  }{% else
    \setlength{\parindent}{0pt}
    \setlength{\parskip}{6pt plus 2pt minus 1pt}}
}{% if KOMA class
  \KOMAoptions{parskip=half}}
\makeatother
\usepackage{xcolor}
\IfFileExists{xurl.sty}{\usepackage{xurl}}{} % add URL line breaks if available
\IfFileExists{bookmark.sty}{\usepackage{bookmark}}{\usepackage{hyperref}}
\hypersetup{
  pdftitle={Clinical Inquiry},
  pdfauthor={Sean Collins},
  hidelinks,
  pdfcreator={LaTeX via pandoc}}
\urlstyle{same} % disable monospaced font for URLs
\usepackage{longtable,booktabs,array}
\usepackage{calc} % for calculating minipage widths
% Correct order of tables after \paragraph or \subparagraph
\usepackage{etoolbox}
\makeatletter
\patchcmd\longtable{\par}{\if@noskipsec\mbox{}\fi\par}{}{}
\makeatother
% Allow footnotes in longtable head/foot
\IfFileExists{footnotehyper.sty}{\usepackage{footnotehyper}}{\usepackage{footnote}}
\makesavenoteenv{longtable}
\usepackage{graphicx}
\makeatletter
\def\maxwidth{\ifdim\Gin@nat@width>\linewidth\linewidth\else\Gin@nat@width\fi}
\def\maxheight{\ifdim\Gin@nat@height>\textheight\textheight\else\Gin@nat@height\fi}
\makeatother
% Scale images if necessary, so that they will not overflow the page
% margins by default, and it is still possible to overwrite the defaults
% using explicit options in \includegraphics[width, height, ...]{}
\setkeys{Gin}{width=\maxwidth,height=\maxheight,keepaspectratio}
% Set default figure placement to htbp
\makeatletter
\def\fps@figure{htbp}
\makeatother
\setlength{\emergencystretch}{3em} % prevent overfull lines
\providecommand{\tightlist}{%
  \setlength{\itemsep}{0pt}\setlength{\parskip}{0pt}}
\setcounter{secnumdepth}{5}
\usepackage{booktabs}
\usepackage{amsthm}
\makeatletter
\def\thm@space@setup{%
  \thm@preskip=8pt plus 2pt minus 4pt
  \thm@postskip=\thm@preskip
}
\makeatother
\ifluatex
  \usepackage{selnolig}  % disable illegal ligatures
\fi
\usepackage[]{natbib}
\bibliographystyle{apalike}

\title{Clinical Inquiry}
\author{Sean Collins}
\date{2022-04-03}

\begin{document}
\maketitle

{
\setcounter{tocdepth}{1}
\tableofcontents
}
\hypertarget{preface}{%
\chapter*{Preface}\label{preface}}
\addcontentsline{toc}{chapter}{Preface}

Added on iOS using Working Copy app as a trial

Clinical Inquiry is a book writing project that aims to provide a book for the Clinical Inquiry course sequence (I, II, III) in the Plymouth State University DPT program. Writing occurs intermittently and largely when the author is teaching the courses in the sequence when he is working on the concepts with students and providing them with notes to supplement class discussions.

Clinical Inquiry as a whole is an explicit combination of what other programs would consider courses in evidence based practice, research methods, statistics, clinical epidemiology (to name a few), and clinical reasoning. The foundational premise is that evidence based practice without clinical reasoning, or clinical reasoning without evidence based practice is incomplete. The terminology ``Clinical Inquiry'' is a mashup up ``Clinical Practice'' and ``Philosophical Inquiry''. As a system of reasoning about and for practice it is founded on the belief that it is necessary to develop theory and practice in conjunction not in isolation. Thinking about these topics has been a career long endeavor, writing about it started in 2015 as a blog by the author called Knowledge Based Practice: Cause, Models and Inference. Between 2015 and 2020 the ideas were put into practice as a Doctor of Physical Therapy (DPT) program at Plymouth State University (PSU). In its current form it is a book in development. As a system Clinical Inquiry aims to provide support for intellectual development and better communication of knowledge for practice.

\hypertarget{about-the-dpt-program-at-psu}{%
\section{About the DPT program at PSU}\label{about-the-dpt-program-at-psu}}

The DPT program at PSU includes a course sequence starting with a course called Clinical Inquiry I which covers the tools that support clinical reasoning and its development such as the connections between generating and utilizing knowledge in practice. These concepts are threaded through several learning experiences that are intended to help students learn principles of clinical reasoning (logical and epistemological foundations). Examples of these experiences include encoding the causal structure of relevant clinical variables in a causal model, using algorithms and broad concepts to guide an evaluation, and intentionally reflecting on clinical experiences as stories that have a causal structure.

\hypertarget{intro}{%
\chapter*{Introduction}\label{intro}}
\addcontentsline{toc}{chapter}{Introduction}

Clinical Inquiry is a system of clinical logic, epistemology and reasoning (CLEaR). The purpose of Clinical Inquiry started off to answer the question: What can be done to improve the utilization of research by physical therapists in practice? {[}1, 2{]} It has now expanded to include explicit consideration of what can be done to improve clinical reasoning through the utilization of knowledge in practice. Knowledge is defined as justified true belief. Goals include increasing the effectiveness of inquiry to: generate knowledge through the rigorous rational (mental, mindful, reflective) interpretation of observations (structured as research, and experienced in practice); and utilize knowledge in both practice and continued research. It is proposed that even by making progress toward these goals we can better educate students in preparation for entry level and subsequently expert practice, reduce cognitive biases and errors in reasoning, and reduce variability in practice decisions in a way that clinical practice guidelines have not been able to achieve.

Clinical Inquiry as a system includes clinical reasoning so it is helpful to define clinical reasoning. A complete understanding of this definition requires an understanding of inference (Chapter 3) so the reader is asked to keep this in mind while progressing through Chapters 1 and 2. Clinical reasoning includes making inferences iteratively across time (dynamic inference); and making decisions based on those inferences. Therefore, clinical reasoning is dynamic inference and decision making. Clinical reasoning as a stream of dynamic inferences that enable clinical decision making are then combined with carrying out the decisions and in sum is called clinical practice. Therefore, clinical practice is combining decisions with carrying out those decisions.

\hypertarget{part1}{%
\chapter*{Part I}\label{part1}}
\addcontentsline{toc}{chapter}{Part I}

\hypertarget{fallacies-bias}{%
\chapter{Fallacies \& Bias}\label{fallacies-bias}}

\hypertarget{inference-decision-making}{%
\chapter{Inference \& Decision Making}\label{inference-decision-making}}

Inference is something inferred or the act of inferring. To infer is to draw a conclusion from a set of facts or premises. Inference is a way of knowing. Our critical realist epistemological axioms provide some starting assumptions and conditions about knowing. Inference provides a way to know. Since reasoning in clinical practice is based on knowing (conclusions drawn from a sets of facts or premises) then clinical reasoning includes the process of making inferences (inferring).

Clinical reasoning includes making inferences and the process is ongoing across time. So clinical reasoning is dynamic inference with decision making. Even the process to decide what fact to collect during an examination is a decision based on a prior inference. Already known facts and prior inferences combine with new facts to generate new inferences which leads to new decisions, and so on. Clinical reasoning is a stream of dynamic inferences that enable clinical decision making which when combined with carrying out the decisions is clinical practice. Clinical practice is combining decisions with carrying out those decisions; and the decisions are made through the dynamic inferences during clinical reasoning.

Dynamic inferences are dynamic in both time and type (temporally and spatially). To understand these dynamic inferences we start with the three foundational types of inference that are dynamically combined during clinical reasoning.

\begin{verbatim}
1.  Deductive inference
2.  Inductive inference
3.  Abductive inference 
\end{verbatim}

\hypertarget{dynamic-inference}{%
\section{Dynamic Inference}\label{dynamic-inference}}

We started this chapter by relating dynamic inference and clinical reasoning. Clinical reasoning is making inferences iteratively in time therefore it is dynamic in time. Clinical inferences are abductive, deductive and inductive and therefore they are dynamic in type. Clinical reasoning is a stream of dynamic inferences that enable clinical decision making. Clinical practice combines decisions with carrying out those decisions. To understand clinical reasoning as dynamic inference the chapter introduced readers to the three foundational types of inference that are combined during clinical reasoning. The reader is now prepared for a deeper consideration of deduction by considering logical theorems in Chapter 4.

\hypertarget{deductive-inference}{%
\section{Deductive Inference}\label{deductive-inference}}

Deductive inference (deduction) is the process of moving from (possibly provisional) acceptance of a set of propositions to a conclusion (which is also a proposition). It is a form of inference you have learned about if you have ever taken an Introduction to Logic course. It is the form of inference covered in Aristotle's syllogisms (which are all valid forms of deductive inference). Deduction has two characteristics that make it unique from among the three types of inference. First, there are valid forms of deductive inference (Chapter 2) where the form is true in all states as well as satisfiable forms where the conclusion is true given a particular set of states of the propositions.{[}1{]} Note that this first characteristic is what truly sets deduction apart from induction and abduction. Second, deductive inference can include a set of universal and specific propositions to generate a universal or specific propositions. Though it is most commonly noted to move from universal and specific propositions to specific conclusion. This second characteristic does not set deduction apart, but when combined with the first characteristic it becomes very powerful in clinical reasoning. The ability to move from universal and specific propositions to specific propositions (or some might say from general to specific, or universals to particulars) is what we do in clinical reasoning when making a treatment decision for a particular patient. Essentially, this is the answer to the question asked by my epidemiology professor in 1997: ``How do you apply sample statistics from a study to a patient?'' The simple answer is: deduction. He was actually asking the question rhetorically, not expecting an answer.
In Chapter 2 the reader was introduced to a valid form of deductive inference with Modus Ponens (3.77):
(3.77) Modus Ponens:\\
Further discussion of (3.77) and other valid and satisfiable expressions of deductive inference are introduced in Chapter 4 utilizing notation from propositional logic.{[}2{]}\\
Examples
With these examples the reader should simply become comfortable with the various ways deductive inference is used generally and in practice.

Example 1
A valid form of deduction with two specific propositions, followed by a specific conclusion:
Bob is taller than John (premise 1)
John is taller than Mike (premise 2)
Therefore Bob is taller than Mike (conclusion)
This example emphasizes the feature of specific propositions to a specific conclusion.

Example 2
A valid form of deduction with a universal and specific proposition, followed by a specific conclusion:
All DPT students are intelligent (premise 1)
Kate is a DPT student (premise 2)
Therefore that Kate is intelligent (conclusion)
This example emphasizes the feature of flowing from universal and specific propositions to a specific conclusion.

Example 3
With valid expressions of deduction, the only way for a conclusion to be false () is for one of the premises to be found untrue.
All swans are white (premise 1)
There is a swan in the other room (premise 2)
Therefore that swan must be white (conclusion)
Here we have a conclusion that might be true or false. But the form of the argument is valid. Despite the validity of the form we know that premise one is false. It could be a black swan. Since premise 1 is false then the conclusion might be false. This example emphasizes the point that all valid expressions of deduction are satisfiable.

Example 4
The use of deductive inference in practice may include the full argument and the desire to achieve the conclusion through manipulation of premises (note that premise two is an intervention).
Respiratory muscle training strengthens respiratory muscles.
Patient does respiratory muscle training.
Therefore the patient strengthens respiratory muscles.
This is a simplification of the thought process going on at the time. But it allows us to notice that with deduction in practice we are usually thinking of the conclusion we want and manipulating one premise. Here we accept the first premise, we want the conclusion to be true, so we intervene with the second premise. That is how most interventions are determined -- some premises are accepted, a conclusion is desired, the premises that are needed to get from the existing premises to the desired conclusion are acted on (intervention). As above we have a conclusion that might be true or false. But the form of the argument is valid so that if the premises are true then the conclusion will be true. This example emphasizes prior points from the examples above (universal and specific premises to specific conclusions; and all valid expressions are satisfiable) and introduces a typical pattern of using deduction in practice.

Example 5
In this example we expand on the above example and sneak in an assumption (unstated premise).
Premises:
Respiratory muscle training strengthens respiratory muscles.
Weak respiratory muscles cause dyspnea on exertion (DOE).
This patient has weak respiratory muscles.
This patient has DOE.
Patient does respiratory muscle training. (Intervention)
Desired conclusions:
This patient no longer has weak respiratory muscles.
This patient no longer has DOE.
We know that not every patient with weak respiratory muscles and DOE that participates in respiratory muscle training has a complete amelioration of DOE. So there are issues. But why does reality not match our conclusion all the time? Is it that this is not a logical form? Or it is that one of the universal premises is not always true and therefore we observe situations where one of the premises is false? Or did we sneak in an assumption? While weak respiratory muscles can cause DOE; there are other causes of dyspnea on exertion. So it is possible for the first conclusion to be true, but the second conclusion to be false. It turns out we have snuck in an assumption (unstated premise) that is actually a fallacy. We have an assumption that equates:
Weak respiratory muscles cause DOE, with
No Weak respiratory muscles cause No DOE
But these are not equivalent (as you will learn in Chapter 4). So we have actually changed the premise when we made our conclusion. We switched from Weakness causes DOE to No Weakness causes no DOE and that is an error. However, not being equivalent does not mean it is automatically untrue. Just that the truth of is not determined (assured) by the truth of the original implication. This example emphasizes different ways deduction can go wrong.
Summary
Hopefully the reader has an understanding and some insights into the use of deduction in clinical reasoning and the possible use of logic to identify errors in clinical reasoning. Systematically breaking down clinical reasoning to identify the logical expressions improves practice and improves research by highlighting what we know and what need to know. Thinking systematically helps us recognize the flaws (formal or informal) in our reasoning. There is a problem being identified with the proliferation of articles on the lack of adoption of ``evidence based practice'' by clinicians and the lack of incorporation of clinical practice guidelines. Clinical Inquiry offers a solution, but it is not being developed as a solution. It is being developed as a better system of thought, better as in a stronger foundation (critical realism over empiricism), and more encompassing of the full spectrum of clinical reasoning (dynamic inference, with deduction as a clear and previously under emphasized component).{[}3{]}

{[}1{]} There are forms categorized as logical fallacies (formal fallacies). With a formal fallacy the premises do not entail (require) the conclusion.
{[}2{]} The truth of inductively determined premises such as p implies q is probable. In Part II on applications we incorporate the probabilistic logic of Adams, Ernest W. ``A primer of probability logic.'' (1996) as an important part of dynamic inference.
{[}3{]} By under emphasized the author means that deduction, while always included in books on clinical reasoning and evidence based practice, is not covered based on its logical foundations. At the same time logic is not a commonly required pre-requisite course for physical therapy programs.

\hypertarget{inductive-inference}{%
\section{Inductive Inference}\label{inductive-inference}}

Inductive inference (induction) is the process of moving from acceptance of a set of premises to a conclusion while allowing for the possibility that the conclusion is false despite all of the premises being true. It is a form of inference you have learned about if you have ever taken a statistics (statistical inference) or research methods course. Induction is unique from deduction because there are no valid expressions of inductive inference where the expression is true in all states.{[}1{]} The common utilization of inductive inference generally proceeds from a set of specific propositions to generate a universal proposition (or a set of specific facts to generate a universal claim about those facts). This is also contrary to the most commonly noted use of deductive inference (which tends to be utilized to proceed from universal and specific propositions to specific conclusion). The ability to move from specific to universal (from particulars to universals) is what we do in research when setting up opportunities to structure our observations (particulars) in an effort to conclude something about reality (universals). Inductive inferences generate the universal premises to be utilized in clinical reasoning when making a treatment decision for a particular patient that actually utilizes deduction. To return to my epidemiology professor in 1997: ``How do you apply sample statistics from a study to a patient?'' The expanded answer is: your sample statistics inductively generate universals from which we can deduce particulars.
Inductive inference can be structured which we will refer to as formal (research based observations and a hierarchy that reflects our belief in the likelihood of true inductive inferences (that is true universals). It can also unstructured which we will refer to as informal (experience observing your surroundings and learning from it). In both formal and informal inductive inferences there are assumptions, methodological considerations that influence the observations, statistical assumptions, influences of prior knowledge, and different levels of consistency of observations. Inductive inference enables a framework for a Clinical Inquiry to compare and contrast the pros and cons of research studies and their derivatives such as systematic reviews with other types of observation such as a clinician's experience and case studies or series.
In Chapter 2 the reader was introduced to a form of inductive inference with a quantification (2.1) that is a constant conjunction of events (observable events):
(2.1) Quantification:
Further discussion of (2.1) and other expressions of inductive inference are introduced in Chapter 5 utilizing axioms and notation from Probability.

\hypertarget{examples}{%
\subsection{Examples}\label{examples}}

With these examples the reader should simply become comfortable with the various ways inductive inference is used generally and in practice.

\hypertarget{example-1}{%
\subsubsection{Example 1}\label{example-1}}

A inductive inference with two specific propositions, followed by a universal conclusion:
A study samples 10 swans from a pond in Massachusetts
All the swans were white
Therefore, all swans are white.
As an inductive inference we leave open the possibility that this conclusion is wrong despite knowing that the premises are true. The presence of one white swan refutes the conclusion.
This example emphasizes the feature of specific propositions based on observation to a universal conclusion.

\hypertarget{example-2}{%
\subsubsection{Example 2}\label{example-2}}

A inductive inference with two specific propositions, followed by a universal conclusion:
A study samples 1000 swans from North America
All the swans were white
Therefore, all swans are white.
As an inductive inference we leave open the possibility that this conclusion is wrong despite knowing that the premises are true but we are more likely to believe this conclusion because of the larger and more representative sample. This example emphasizes that the number of observations and from where they are sampled changes our believe in the conclusion.

\hypertarget{example-3}{%
\subsubsection{Example 3}\label{example-3}}

An inference with two specific propositions, followed by a specific conclusion:
A study samples 10 swans from a pond in Massachusetts
All the swans were white
Therefore, these 10 swans are white.
This is actually a deductive inference snuck in to prove a point. Here we do not leave open the possibility that this conclusion is wrong despite knowing that the premises are true. If the premises are true then the conclusion is true. The conclusion's scope is so limited the only way it would be wrong is if the 10 swans were not all white. The only way this is wrong is if there is a problem with measurement of swan color (basic reliability of sensory perception).
This example emphasizes the feature of specific propositions based on observation to a specific conclusions of having limited contribution outside of the particular scope.

\hypertarget{example-4}{%
\subsubsection{Example 4}\label{example-4}}

A inductive inference with two specific propositions, followed by a universal conclusion:
A study randomly samples swans across the Northern Hemisphere
swans were white, swans were black
Therefore, 10\% of swans are black
As an inductive inference we leave open the possibility that this conclusion is wrong despite knowing that the premises are true but we are more likely to believe this conclusion because of the larger and more representative sample. Using statistical inference (covered in Part II after having a foundation in Probability in Chapter 5) we can determine the probability of future samples of swans also having 10\% of them being black as a confidence interval. The size of the confidence interval is related to how many swans are sampled. And the universality of this conclusion is also related to where it was sampled. For example, there may be reasons for a different color pattern in the Southern Hemisphere. This example emphasizes that the number of observations and from where they are sampled changes our believe in the conclusion.

It may happen because of randomness, that is one category of error leading to inductive problems. Essentially, the methods of observation, and analysis should be attempts to minimize the inductive problem. Whenever reading a paper we can considering it's ``external validity'' (also called generalizability) by considering the likelihood that the study has an induction problem (will be refuted in some case).

\hypertarget{summary}{%
\subsection{Summary}\label{summary}}

Hopefully the reader has an understanding and some insights into the use of induction in clinical reasoning. There is value in defining induction based on its differences with deduction. Furthermore there is value in defining both research (including its derivatives) and clinical experience based on their foundations in inductive inference. By having a common framework such as inductive inference we can compare and contrast the universals generated through these different approaches to learning. Finally, there is value in students and professionals looking to advance practice to consider the inductive inferences utilized to generate the universals they utilizing alongside the deductive inferences they are using to generate particulars in practice. The dynamic inference of Clinical Inquiry relies on the interplay between inductively generated universal conclusions and deductively generated particular conclusions.

{[}1{]} There can be valid expressions about induction that are actually deductive inferences.

\hypertarget{abductive-inference}{%
\section{Abductive Inference}\label{abductive-inference}}

Abductive inference (abduction) adopts characteristics from both deduction and induction. The characteristic from deduction is that abduction tends to progress from a acceptance of a set of universal and particular premises to a particular conclusion. A characteristic from induction is that abduction allows for the possibility that the conclusion is false despite all of the premises being true. It is a form of inference you may never have heard about before by name. However, if you have ever inferred the possible reasons for a current situation you have used abductive inference. For example, if you have ever inferred that it rained over night based on seeing everything wet and puddles outside in the morning then you have used abductive inference. When Sherlock Holmes says to Watson, ``It's deduction my dear Watson'' he really should have been saying ``It's abduction my dear Watson.'' It is just that the terminology and distinguishing features of abduction had not been utilized yet when Sir Arthur Conan Doyle was writing.
If you have taken a logic course you may have learned about abduction as a formal fallacy called affirming the consequent. Since you know about Modus Ponens affirming the consequent is easy to understand:

(3.77) Modus Ponens:\\
Affirming the consequent:

As you can see, Affirming the consequent says that if p implies q, and you have q then it implies p.~However, this is not the case all the time and therefore this is not valid (not true in all states). We will leave it to the reader - as an Exercise - to determine whether Affirming the consequent is satisfiable.
Abduction was introduced in the writings C.S. Peirce around the beginning of the 20th century.{[}1{]} He was interested in how scientists (he was a geologist prior to becoming a logician and philosopher) came up with hypotheses to test experimentally. He recognized that hypotheses are generated by trying to figure out what is behind (underlying, causing\ldots) our observations. This type of thinking is so pervasive in our every day and professional lives it remains a challenge{[}2{]} to recognize when it is being done and to consider the implications of the differences between deduction and induction. Sherlock Holmes was most likely too correct too often given the fallacy underlying his deductions and as we will learn later given the influence that probability has on abduction (as it does on induction).
Abduction is a critical part of clinical reasoning for practice. Patient's seek physical therapy because they have a problem. The solution to a problem cannot begin unless the problem is understood. Understanding the problem requires abduction. But abduction has limitations, and therefore understanding the problems has limitations. To improve practice we must understand how to utilize abduction with due consideration its limitations.

\hypertarget{examples-1}{%
\subsection{Examples}\label{examples-1}}

\hypertarget{example-1-1}{%
\subsubsection{Example 1}\label{example-1-1}}

An abductive inference with a universal and specific proposition, followed by a specific conclusion:
Heart failure (HF) leads to dyspnea on exertion (DOE)
A person has DOE
Therefore, the person has HF
As an abductive inference we leave open the possibility that this conclusion is wrong despite knowing that the premises are true. The presence of DOE does not mean, conclusively, that the person has HF. There are other possible explanations.
This example emphasizes the most important feature abduction that we don't actually know whether the person has HF or not, but it raises the question of whether they have HF and therefore warrants investigation. As Peirce had intended, it raises the hypothesis that needs to be tested. This example also emphasizes what you need to know from physiology and pathophysiology, that is, that HF leads to DOE, and also what else leads to DOE?

\hypertarget{example-2-1}{%
\subsubsection{Example 2}\label{example-2-1}}

An abductive inference with the same form as above:
Bicep tendonitis leads to pain with elbow flexion
A person has pain with elbow flexion
Therefore, the person has bicep tendonitis
This example emphasizes the same point as above in a different context. This example also emphasizes the examination process.

\hypertarget{example-3-1}{%
\subsubsection{Example 3}\label{example-3-1}}

An abductive inference with the same form as above:
A weak gluteus medius muscle causes trendelenberg gait
A person has trendelenberg gait
Therefore, the person has a weak gluteus medius muscle
This example emphasizes the same point as above in a different context. This example also emphasizes the importance of knowing anatomy (muscle attachments, actions, innervation), mechanics and gait pattern recognition.

\hypertarget{summary-1}{%
\subsection{Summary}\label{summary-1}}

Hopefully the reader has an understanding and some insights into the use of abduction in clinical reasoning. There is value in defining abduction based on its differences and similarities with deduction and induction. A key reason to understand abduction is to understand its limitations as the examples above emphasize and due to the importance of its use in clinical reasoning.
Abductive inference has been intentionally presented and examples have emphasized its most general form. When we learn about probability in Chapter 5 we connect abduction to Bayes theorem and consider as Bayes inference as a special form of abduction. Then in Part II on applications we use Bayes inference to understand tests of diagnostic accuracy that you may be familiar with such as sensitivity, specificity and likelihood ratios. Combined with our causal models from Chapter 7 we then consider clinical reasoning when making a differential diagnosis which combines the hypothesis generating property of abduction with the diagnostic accuracy considerations of Bayes inference. Having both the general form of abduction and understanding its connection with Bayes theorem, Bayes inference and differential diagnosis allows Clinical Inquiry to offer a robust approach to considering what must be known and how to organize that knowledge for the abductive component of dynamic inference.

{[}1{]} Shanahan, Timothy. ``The first moment of scientific inquiry: CS Peirce on the logic of abduction.'' Transactions of the Charles S. Peirce Society 22, no. 4 (1986): 449-466.
{[}2{]} Writing this reminds me of my favorite bench to sit on during my 17 years of teaching at UMass Lowell in the south campus quad. It had a question that resonates with the challenges of those things most pervasive. ``Does a fish know that it is wet?''

\hypertarget{philosophical-inquiry-of-practice}{%
\chapter{Philosophical Inquiry of Practice}\label{philosophical-inquiry-of-practice}}

\hypertarget{causation-models-stories}{%
\chapter{Causation, Models \& Stories}\label{causation-models-stories}}

\hypertarget{probability}{%
\chapter{Probability}\label{probability}}

\hypertarget{probability-introduction}{%
\section{Probability introduction}\label{probability-introduction}}

\hypertarget{state-space-sample-space}{%
\subsection{State space (sample space)}\label{state-space-sample-space}}

An event space is often denoted by sigma (\(\Sigma\)) and contains all sets of outcomes.

A sample space of an experiment is denoted by omega (\(\Omega\)) and contains all possible outcomes.

We will usually be dealing with \(\Omega\), but for something like a 6 sided dice we can use \(\Sigma\).

If \(E\) defines an event, (say rolling a 6 on a dice), then \{1, 2, 3, 4, 5, 6\} is the event space (all sets of outcomes)

\(P(E) = \frac{E}{\Sigma}\)

\(P(rolling \space a\space 6) = \frac{rolling \space a\space 6}{6} = \frac{1}{6}\)

You expect a 6 about \(\frac{1}{6}\). What is an unexpected outcome?

\hypertarget{probability-is-bound-between-0-and-1}{%
\subsection{Probability is bound between 0 and 1}\label{probability-is-bound-between-0-and-1}}

\(0 \le P(E) \le 1\)

\hypertarget{certainty-and-uncertainty}{%
\subsection{Certainty and uncertainty}\label{certainty-and-uncertainty}}

Certainty includes \(P(E) = 0\) and \(P(E) = 1\)

Therefore the probability of an uncertain event is \(0 < P(E) < 1\) (particular)

Therefore the probability of all uncertain events are \(0 < P(E) < 1\) (universal)

\hypertarget{connection-to-deductive-logic}{%
\subsection{Connection to deductive logic}\label{connection-to-deductive-logic}}

If \(E\) is \(TRUE\), then \(P(E) = 1\)

If \(E\) is \(FALSE\), then \(P(E) = 0\)

\hypertarget{using-sets-as-probability-spaces}{%
\subsection{Using Sets as ``Probability Spaces''}\label{using-sets-as-probability-spaces}}

Recall:

\begin{itemize}
\tightlist
\item
  \(E \subseteq F\) is that E is a ``subset'' of F
\item
  \(E \cap F\) is set intersection and is similar to logical ``and'' (\(\land\))
\item
  \(E \cup F\) is set union and is similar to logical ``or'' (\(\lor\))
\end{itemize}

If \(E \subseteq F\), then \(P(E) \le P(F)\)

If \(E \cap F\), then \(P(E \lor F) = P(E)+P(F)\)

\hypertarget{conditional-probability}{%
\subsection{Conditional probability}\label{conditional-probability}}

\(P(E \mid F)\) is the \(P(E)\) ``given that'' \(P(F) = 1\) or you could say \(F\) is \(TRUE\)

Therefore:

If \(E \subseteq F\), then \(P(F \mid E) = 1\)

If \(E \cap F = 0\), then \(P(E \mid F) = 0\)

\hypertarget{causation-connections}{%
\subsection{Causation connections}\label{causation-connections}}

If \(E\) causes \(F\) we can use the notation \(E \rightarrow F\);

Then it is true that \(P(F \mid E) \ge P(F \mid \neg E)\) (note that \(\neg\) means ``not'')

This is the notation for Hill's criteria, ``strength of association'' and is consistent with the statement: ``causation implies correlation'', but notice that it is inconsistent with the statement ``correlation implies causation'' (as you have learned, correlation does not imply causation).

\hypertarget{notes-and-thoughts-on-chapter-3-probability-in-the-clinical-encounter}{%
\section{Notes and thoughts on Chapter 3: Probability in the clinical encounter}\label{notes-and-thoughts-on-chapter-3-probability-in-the-clinical-encounter}}

Notes and thoughts on Chapter 3 from the book: (Anjum, R. L., Copeland, S., \& Rocca, E. (2020). Rethinking causality, complexity and evidence for the unique patient: a CauseHealth Resource for healthcare professionals and the clinical encounter (p.~241). Springer Nature.){[}\url{https://link.springer.com/book/10.1007\%2F978-3-030-41239-5}{]}

\hypertarget{uncertainty-and-probability-in-the-single-case}{%
\subsection{3.1 Uncertainty and Probability in the Single Case}\label{uncertainty-and-probability-in-the-single-case}}

A goal of this book - and indeed of all clinical inquiry - is to consider the probability that intervention \(X\) will work this time for this patient

Note that the probability that \(X\) ``will work'' this time for this patient is a conditional probability (let's take \(P(X)\) to be the probability that intervention \(X\) works).

Therefore, \(P(X \mid p_i)\), is the probability that \(X\) will occur (and \(X\) is that intervention \(X\) will work) given \(p_i\) (note that I'm using \(p_i\) as notation for ``this time for this patient'').

For one moment - consider what it means to say an intervention works? It means that an outcome has been achieved. If you have a headache, and you take an Tylenol capsule, you consider that it works if your headache goes away. That is an outcome indicating the intervention worked.

We can also break \(X\) into a conditional probability, since \(P(X)\) is the probability that intervention \(I\) is effective; and since \(I\) being effective depends on \(O\), the outcome:

\(P(X) = P(O \mid I)\) (the probability of \(X\) being effective is the probability of \(O\) given \(I\))

Notice that this conditional probability \(P(O \mid I)\) is equivalent to the causal claim: \(I \rightarrow O\)

\hypertarget{where-does-this-probability-come-from-what-does-it-represent-what-do-we-mean-by-it}{%
\subsubsection{Where does this probability come from? What does it represent? What do we mean by it?}\label{where-does-this-probability-come-from-what-does-it-represent-what-do-we-mean-by-it}}

\hypertarget{probability-from-statistics-frequentism}{%
\subsection{3.2 Probability from Statistics: Frequentism}\label{probability-from-statistics-frequentism}}

Frequentism is the probability that you're familiar with from statistics. It is considered ``objective'' in that it is based on measurements of observations. These probabilities are often estimated from samples in an attempt to understand something about the population. This is the process of statistical inference we'll talk about soon, and is ``induction'' as we've already discussed.

The assumption is that as the sample size (\(n\)) gets larger we are more confident in the estimate:

As \(n_{sample}\) approaches \(n_{population}\) we are more confident in the estimate of the population (and thus maybe ``universal'') probability.

We'll talk a lot more next week about confidence in these estimates when we discuss statistical inference using ``confidence intervals''.

Note that estimates are both variable and uncertain. Variability is part of their behavior. In fact, variability says something about the events.

\hypertarget{evidence-based-approaches}{%
\subsubsection{3.2.1 Evidence based approaches:}\label{evidence-based-approaches}}

Usually rely on estimates described from samples. The issue, and challenge, is applying these estimates to apatient that was not part of that sample.

\hypertarget{randomization-inclusion-and-exclusion-criteria-in-population-sample-trials}{%
\subsubsection{3.2.2 Randomization, Inclusion and Exclusion Criteria in population (sample) trials:}\label{randomization-inclusion-and-exclusion-criteria-in-population-sample-trials}}

Used to reduce bias.

Bias associated with unknown confounders (randomization)

Bias associated with known confounders (inclusion and exclusion criteria)

\hypertarget{internal-and-external-validity-of-causal-claims-from-rcts}{%
\subsubsection{Internal and external validity of causal claims from RCTs:}\label{internal-and-external-validity-of-causal-claims-from-rcts}}

Causal claims from RCTs are statements of whether some intervention is effective (\(X\)).

\textbf{Internal Validity}

If \(X\) is determined to be \(TRUE\) for a sample, we can consider \(X \subseteq S\) which means \(X\) as a characteristic is one of the characteristics of the sample \(S\).

Threats to internal validity impact whether the study actually demonstrated what they claim to have demonstrated.

\textbf{External Validity}

Then we need to consider whether the sample is a subset of the population (whether the characteristics of the sample are true for the population: \(S \subseteq P\)

Threats to external validity impact whether the findings of the study in this sample, apply to the population (related to statistical inference and induction)

Overall: \((X \subseteq S) \land (S \subseteq P) \Rightarrow (X \subseteq P)\)

Which means that if X is true for the sample, and the sample reflects the population, then X is true for the population (transitivity).

The mission of this book, and the project from which is comes (CauseHealth), and my work for the past 17 years or so, has been dealing with the fact that the patient you're working with (\(p_i\)) is not in the sample, and not necessarily in the population.

Just because: \((X \subseteq S) \land (S \subseteq P) \Rightarrow (X \subseteq P)\) is true (and that is usually uncertain)

It does not mean that:

\((X \subseteq P) \land (p_i \subseteq P) \Rightarrow (X \subseteq p_i)\)

In fact:

\((X \subseteq P) \land (p_i \subseteq P) \nRightarrow (X \subseteq p_i)\)

\hypertarget{probability-as-degree-of-belief---subjective-credence}{%
\subsection{3.3 Probability as Degree of Belief - Subjective Credence}\label{probability-as-degree-of-belief---subjective-credence}}

Three things:
First, don't conflate conditional probabilities with ``subjective''
Second, don't conflate Bayesian probabilities or inference with ``subjective''
Third, don't make as big a deal of ``objective'' and ``subjective'' probabilities as they do in this chapter.

As you will come to appreciate, all objective probabilities are obtained with some subjectivity; and all subjective probabilities are founded in some sort of objectivity.

You'll also come to appreciate that all causal claims can be considered as conditional probability; and since all clinical practice is based on causal claims, then all clinical practice is based on conditional probabilities. Every question you ask, every action you take is because whatever you know ``right now'' is adjusted based on ``what you have just learned.'' Meaning, you're always adjusting your belief / understanding / knowledge based on new information - - which is the same thing as saying you're always using conditional probabilities.

\hypertarget{updating-belief---as-stated-above-absolutely-unavoidable-if-you-want-to-be-a-good-clinician}{%
\subsubsection{3.3.1 Updating belief - as stated above, absolutely unavoidable if you want to be a good clinician}\label{updating-belief---as-stated-above-absolutely-unavoidable-if-you-want-to-be-a-good-clinician}}

\hypertarget{understanding-the-basic-bayesian-formula---to-be-covered-in-chapter-7}{%
\subsubsection{3.3.2 Understanding the basic Bayesian formula - to be covered in Chapter 7}\label{understanding-the-basic-bayesian-formula---to-be-covered-in-chapter-7}}

\hypertarget{uncertainty-as-a-lack-of-knowledge}{%
\subsubsection{3.3.3 Uncertainty as a lack of knowledge}\label{uncertainty-as-a-lack-of-knowledge}}

Here I believe they are overstating the difference between ontology and epistemology.

There is variability in an event \(E\), we try to figure out why, but whether it is ontological or epistemological may not be reconciled; this does not change the value of knowing what we can find out.

\hypertarget{probability-as-dispositional-and-intrinsic-properties}{%
\subsection{3.4 Probability as Dispositional and Intrinsic Properties}\label{probability-as-dispositional-and-intrinsic-properties}}

Provides some language - I'm honestly not yet sure how it benefits the overall goal.

For example, I think we've always thought of probabilities are reflection the ``propensity'' of an event. Some of that is ``intrinsic'' to the characteristics of the event itself, some of it is due to variability of other factors.

Once again we see that the purpose of the book includes a conditional probability: ``The individual propensities of a single patient, treatment or context are given by it's unique combination of dispositions and the dispositions' degree of tendency (propensity) toward particular outcomes in that individual case.''

\(P((O \mid I) \mid p_i)\)

The probability of an outcome given the intervention, given this time for this patient.

\hypertarget{propensities-and-the-clinic}{%
\subsection{3.5 Propensities and the Clinic}\label{propensities-and-the-clinic}}

Useful with consideration to the notes above. How usual is yet to be determined (at least in my opinion).

\hypertarget{statistical-inference-induction}{%
\chapter{Statistical Inference (Induction)}\label{statistical-inference-induction}}

This chapter describes the use Confidence Intervals as a tool for making statistical inferences. Statistical inferences use probability to consider how effective inductive inferences from a sample are at providing an an estimate of a characteristic (often referred to as a parameter) in a population. Samples are collections of individual (particular) observations and a population represents a universal (or general) characteristic. Estimates from a sample are called statistics, whereas estimates from a population are parameters.

Two common parameters that we attempt to estimate are proportions (percentage of some characteristic) and means (mathematical average). The actual numerical value of a population proportion or mean would rarely be known.

In order to estimate the population proportion or mean we calculate a sample proportion or mean from a sample. The value of the sample mean is then used to estimate an unknown population mean.

Since we know that there is error in our estimate, the question becomes how much error? The amount of error is related to the size of the sample.

\hypertarget{confidence-intervals-ci}{%
\section{Confidence Intervals (CI)}\label{confidence-intervals-ci}}

A confidence interval provides an interval around a sample estimate that contains the population parameter with a given probability. For example, a 95\% CI is an interval that is expected to contain the true population parameter 95\% of the time.

If the mean height of a sample of students is 69 inches, and the 95\% CI = {[}66, 72{]}, then the population parameter (mean height) has a probability of 0.95 to between 66 and 72 inches.

Another interpretation is that if we were to randomly sample 100 samples of the same size from the same population, 95\% of the sample means would fall within the ranage of the 95\% CI = {[}66, 72{]}.

\hypertarget{general-equations}{%
\subsection{General equations}\label{general-equations}}

\(sample \space estimate = population \space parameter + random \space error\)

\(95\%CI=sample \space estimate \pm z \times \space standard \space error\)

Where \(z\) is a multiplier that comes from the normal curve. For a 95\% CI \(z\) is often taken to be 1.96 (or 2 more conservatively). Another option is to use the \(t\) distribution (flatter and wider than the normal curve) which is recommended for samples of less than 30.

\hypertarget{ci-for-a-proportion}{%
\subsection{CI for a Proportion}\label{ci-for-a-proportion}}

\(\sqrt{\frac{}{}}\)
\(95\%CI=sample \space proportion \pm z \times \sqrt{\frac{sample \space proportion(1-sample \space proportion)}{n}}\)

Note the Standard Error of a sample proportion is:

\(Standard \space Error \space of \space sample \space proportion=\sqrt{\frac{sample \space proportion(1-sample \space proportion)}{n}}\)

\hypertarget{ci-for-a-mean}{%
\subsection{CI for a Mean}\label{ci-for-a-mean}}

\(95\%CI=sample \space mean \pm z \times \frac{standard \space deviation}{\sqrt{n}}\)

Note the Standard Error of the Mean is:

\(Standard \space Error \space of \space the \space Mean=\frac{standard \space deviation}{\sqrt{n}}\)

\hypertarget{ci-to-evaluate-differences}{%
\subsection{CI to evaluate differences}\label{ci-to-evaluate-differences}}

Standard Error (SE) of a Difference:

\(SE \space for \space a \space Difference=\sqrt{(SE_1)^2 +(SE_2)^2}\)

Note: \(SE_1\) is the SE for group 1; and \(SE_2\) is the SE for group 2.

\(95\%CI=difference \space between \space samples \pm z \times (SE \space for \space a \space Difference)\)

\hypertarget{bayesian-inference}{%
\chapter{Bayesian Inference}\label{bayesian-inference}}

\hypertarget{preliminaries}{%
\section{Preliminaries}\label{preliminaries}}

\hypertarget{contingency-table}{%
\subsection{Contingency Table}\label{contingency-table}}

\begin{longtable}[]{@{}ccc@{}}
\toprule
& \(D\) & \(\neg D\) \\
\midrule
\endhead
\(S\) & True Positive (TP) (a) & False Positive (FP) (b) \\
\(\neg S\) & False Negative (FN) (c) & True Negative (TN) (d) \\
\bottomrule
\end{longtable}

\hypertarget{sensitivity-specificity}{%
\subsection{Sensitivity / Specificity}\label{sensitivity-specificity}}

Sensitivity: \(P(S|D)=\dfrac{a}{a+c}\) (Rate of True Positives)

Specificity: \(P(\neg S|\neg D)=\dfrac{d}{b+d}\) (Rate of True Negatives)

1 - Sensitivity: \(P(\neg S|D)=\dfrac{c}{a+c}\) (Rate of False Negatives)

1 - Specificity: \(P(S|\neg D)=\dfrac{b}{b+d}\) (Rate of False Positives)

\hypertarget{positive-likelihood-ratio}{%
\subsection{Positive Likelihood Ratio}\label{positive-likelihood-ratio}}

\(+LR=\dfrac{Sensitivity}{1-Specificity}\)

\(+LR=\dfrac{P(S|D)}{P(S|\neg D)}\)

\(+LR= \dfrac{P(TP)}{P(FP)}\)

\(+LR= \dfrac{a(b+d)}{b(a+c)}\)

\hypertarget{negative-likelihood-ratio}{%
\subsection{Negative Likelihood Ratio}\label{negative-likelihood-ratio}}

\(-LR=\dfrac{1-Sensitivity}{Specificity}\)

\(-LR=\dfrac{P(\neg S|D)}{P(\neg S|\neg D)}\)

\(-LR= \dfrac{P(FN)}{P(TN)}\)

\(-LR= \dfrac{c(b+d)}{d(a+c)}\)

\hypertarget{bayes-formula}{%
\section{Bayes Formula}\label{bayes-formula}}

\(P(D|S)=\dfrac{P(S|D)\cdot P(D)}{P(S)}\)

The \(P(D)\) and \(P(S)\) are the ``priors'' - or ``baseline'' probabilities of the disease and the sign

\hypertarget{alternative-format}{%
\subsection{Alternative format}\label{alternative-format}}

\(P(\neg D|\neg S)=\dfrac{P(\neg S|\neg D)\cdot P(\neg D)}{P(\neg S)}\)

The \(P(\neg D)\) and \(P(\neg S)\) are the ``priors'' - or ``baseline'' probabilities of not having the disease or the sign

\hypertarget{probability-logic}{%
\chapter{Probability Logic}\label{probability-logic}}

\hypertarget{complexity}{%
\chapter{Complexity}\label{complexity}}

\hypertarget{stories}{%
\chapter{Stories}\label{stories}}

\hypertarget{part2}{%
\chapter*{Part II}\label{part2}}
\addcontentsline{toc}{chapter}{Part II}

\hypertarget{part3}{%
\chapter*{Part III}\label{part3}}
\addcontentsline{toc}{chapter}{Part III}

  \bibliography{book.bib,packages.bib}

\end{document}
